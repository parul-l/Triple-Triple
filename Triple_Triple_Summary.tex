\documentclass[11pt]{article}

% PACKAGES
\usepackage{amsmath}
\usepackage{latexsym}
\usepackage{amssymb}
\usepackage{ifthen}
\usepackage{xcolor}
\usepackage{hyperref}

% PAGE FORMAT
\oddsidemargin 0.0in
\evensidemargin 0.0in
\textwidth 6.5in
\textheight 8.5in
\topmargin -0.25in

%For commenting out
\newcommand{\junk}[1]{}

% General Commands

\newcommand{\formal}[1]{\ensuremath{\textsf{#1}}}
\newcommand{\imply}{\ensuremath{\Rightarrow   }}
\newcommand{\dimply}{\ensuremath{\Leftrightarrow   }}
\newcommand{\nat}{\ensuremath{  \mathbb{N}   }}
\newcommand{\dword}[1]{\textbf{#1}}
\newcommand{\upint}[1]{\ensuremath{\lceil #1 \rceil}}
\newcommand{\downint}[1]{\ensuremath{\lfloor #1 \rfloor}}


% NEW COMMANDS

\newcommand{\product}{\ensuremath{\pi}}
\newcommand{\seq}[1]{  \ensuremath{ \langle#1 \rangle}  }
\newcommand{\sym}{\ensuremath{ \mathbf{S} }}
\newcommand{\ptype}{\ensuremath{\rho}}

\newcommand{\forward}{\formal{F}}
\newcommand{\back}{\formal{B}}
\newcommand{\dist}{\formal{Dist}}
\newcommand{\norm}{\formal{Norm}}

\newcommand{\FC}{\ensuremath{\mathcal{FC}}}
\newcommand{\ET}{\ensuremath{\mathcal{ET}}}
\newcommand{\VT}{\ensuremath{\mathcal{VT}}}
\newcommand{\FCtoET}{\ensuremath{\mathbf{FCET}}}
\newcommand{\ETtoVT}{\ensuremath{\mathbf{ETVT}}}
\newcommand{\DualMap}{\ensuremath{\formal{Dual}}}
\newcommand{\themap}{\ensuremath{\mathbf{MAP}}}

% THEOREMS

\newtheorem{defn}{Definition}[section]
\newtheorem{thm}[defn]{Theorem}
\newtheorem{prop}[defn]{Proposition}
\newtheorem{cor}[defn]{Corollary}
\newtheorem{conj}[defn]{Conjecture}
\newtheorem{quest}[defn]{Question}
\newtheorem{lemma}[defn]{Lemma}
\newtheorem{notation}[defn]{Notation}
\newtheorem{claim}[defn]{Claim}
\newtheorem{open}{Open Question}
\newtheorem{example}[defn]{Example}
\newtheorem{remark}[defn]{Remark}

% ENVIRONMENTS
\newenvironment{proof}{\noindent \textbf{Proof}  \begin{quote}}{ $\blacksquare$   \end{quote}}
\newenvironment{proofof}[1]{\noindent \textbf{Proof of #1}  \begin{quote}}{ $\blacksquare$   \end{quote}}



%\title {{\bf Real World Applications for MTH 05} \\
%Department of Mathematics and Computer Science}
%\author {Parul Laul, Natalia Novak,  Kerry Ojakian}






\begin{document}
%\maketitle





\noindent
%\begin{center}
{\Large \textbf{Project:} Triple-Triple}\\ \\
\textbf{Date:} Sunday, May 8, 2016\\


%\begin{abstract}  
%\end{abstract}



\section*{Motivation:} 

In basketball, a player's value to a team is generally measured based on his offensive and defensive measurables. In particular, statistics on his scoring ability, such as 2- and 3-point field goal efficiency, free throw percentages, and assists per game averages, provide a relatively suitable measure to gauge his offensive abilities. Similarly, a record of rebounds, blocks, and steals allows a team to make an assessment on what type of defensive skills a player may offer. Clearly, a player with relatively high numbers in both of these categories would appear to be of "high value", and naively, would be considered a top acquisition for any team. However, just a little more thought would convince one that there are a number of other factors involved in choosing the optimal player for a team, the biggest ones being how well he fits with the team's current roster and how much will he cost.  

\section*{Project Description:}
I intend to first study six factors (Triple-Triple!) to determine a player's value to a team: (i) offensive skills, (ii) defensive skills, (iii) endurance, (iv) size, (v) speed, and (iv) price. All six categories themselves have a number of factors that will be important. For parts (i) and (ii), in addition to the obvious measurable statistics listed above, the results of the paper (http://arxiv.org/pdf/1405.0231v2.pdf) give a measure of defensive skill level using player positions at all times on the court. These include analyzing points allowed, changing shot selections, etc. The endurance factor will keep track of a player's average minutes played, his age, and his ability to perform for extended time periods. Size and speed take in to account the different positions on the court, and finally, the price a player refers to his salary.

Using the six factors identified above and data on how well teams play, I plan on identifying free agents with the greatest value added given cost constraints. That is, I intend to propose a function that determines how well a group of players will play together subject to a team salary constraint. Given the money a team is willing to spend, is acquiring player X actually worth it?\\

Continued below...

\newpage
Mathematically, we describe the idea as follows:\\

Let $X$ = \{ NBA players \}.  For every $X_i$ in $X$, we associate six parameters to $X_i$:

$$
X_i \rightarrow
\begin{cases}
\text{offensive ability}
\rightarrow
	\begin{cases}
	\text{shooting} \\
	\text{assists} \\
	\text{getting to rim}\\
	\text{free throw attempts}\\
	\text{turnovers}
	\end{cases}\\
\text{defensive ability}
\rightarrow
	\begin{cases}
	\text{rebounds}\\
	\text{blocks}\\
	\text{steals}\\
	\text{x-factor (hustle, points allowed, switching)}
	\end{cases}\\
\text{endurance}\\
\text{size}\\
\text{speed}\\
\text{price}\\
\end{cases}
$$ 

Other factors to consider:
\begin{itemize}
\item playoff appearances
\item playoff stats
\item number of championships
\item all-star appearances
\end{itemize}

Let $\mathbb{T} = \{ T_i \}$ denote the set of teams, $|\mathbb{T}| = 29$. For every $i$, 
$T_i = \{ X_{i_1}, \cdots X_{i_{n_i}} \}$, where $n_i$ is the number of players on team $T_i$. Let
$S \subset \{ X_{i_1}, \cdots X_{i_{k_i}} \}$ denote a subset of $T_i$, $k_i \leq n_i$.  Define 
$$
price(S) = \sum_{j =1}^{k_i} price(X_{i_j}),
$$
where $X_{i_j} \in S$, and $price(X_{i_j})$ denotes the price of player $X_{i_j}$. The goal is to create a {\em group dynamic function} $g(S)$, that calculates how well the players in $S$ play together.  Specifically, we would like to maximize $g(S)$ subject to a financial constraint, $price (S) \leq \text{{\em some max amount of money}}$.

\section*{Model Construction}
To determine a value of an individual player to a cooperative game, we look to create some sort of {\em Shapely value}. Below we describe suggestions of assessing skill.
\begin{enumerate}
\item Offensive skills.\\
Things to keep in mind here are the ability of a player to create and make open shots, to make open shots off an assist, to make contested shots, to pass when driving. For each player, we find the probability that at every "event" in a play
\begin{itemize}
\item he drives and passes inside
\item he drives and passes on the perimeter
\item he shoots
\item he turns the ball over
\end{itemize}

With this we can simulate plays and determine how well they perform together. For example, we expect a team of 5 point guards to frequently drive and pass, or get blocked often, so that this team scores inefficiently. Similarly, we expect a team of 5 centers to be slow, turn the ball over, all crowded in the paint, etc. 
\end{enumerate}

%%%%%%%%%%%%%%%%%%%%%%%%%%%%%%%%%%%%%%%%%%%%%%%%%%%%%%%%%%%%%%%%%%%%%%%%%%
%%%%%%%%%%%%%%%%%%%%%%%%%%%%%%%%%%%%%%%%%%%%%%%%%%%%%%%%%%%%%%%%%%%%%%%%%%%%%%%%%%%%%%%%%%%%%%%%
%%%%%%%%%%%%%%%%%%%%%%%%%%%%%%%%%%%%%%%%%%%%%%%%%%%%%%%%%%%%%%%%%%%%%%%%%%%%%%%%%%%%%%%%%%%%%%%%
%%%%%%%%%%%%%%%%%%%%%%%%%%%%%%%%%%%%%%%%%%%%%%%%%%%%%%%%%%%%%%%%%%%%%%%%%%%%%%%%%%%%%%%%%%%%%%%%
\begin{thebibliography}{MA}


\bibitem{FMBG} {http://arxiv.org/pdf/1405.0231v2.pdf} \end{thebibliography}
\end{document}
